\documentclass[]{article}
\usepackage{lmodern}
\usepackage{amssymb,amsmath}
\usepackage{ifxetex,ifluatex}
\usepackage{fixltx2e} % provides \textsubscript
\ifnum 0\ifxetex 1\fi\ifluatex 1\fi=0 % if pdftex
  \usepackage[T1]{fontenc}
  \usepackage[utf8]{inputenc}
\else % if luatex or xelatex
  \ifxetex
    \usepackage{mathspec}
  \else
    \usepackage{fontspec}
  \fi
  \defaultfontfeatures{Ligatures=TeX,Scale=MatchLowercase}
\fi
% use upquote if available, for straight quotes in verbatim environments
\IfFileExists{upquote.sty}{\usepackage{upquote}}{}
% use microtype if available
\IfFileExists{microtype.sty}{%
\usepackage{microtype}
\UseMicrotypeSet[protrusion]{basicmath} % disable protrusion for tt fonts
}{}
\usepackage[margin=1in]{geometry}
\usepackage{hyperref}
\hypersetup{unicode=true,
            pdfborder={0 0 0},
            breaklinks=true}
\urlstyle{same}  % don't use monospace font for urls
\usepackage{graphicx,grffile}
\makeatletter
\def\maxwidth{\ifdim\Gin@nat@width>\linewidth\linewidth\else\Gin@nat@width\fi}
\def\maxheight{\ifdim\Gin@nat@height>\textheight\textheight\else\Gin@nat@height\fi}
\makeatother
% Scale images if necessary, so that they will not overflow the page
% margins by default, and it is still possible to overwrite the defaults
% using explicit options in \includegraphics[width, height, ...]{}
\setkeys{Gin}{width=\maxwidth,height=\maxheight,keepaspectratio}
\IfFileExists{parskip.sty}{%
\usepackage{parskip}
}{% else
\setlength{\parindent}{0pt}
\setlength{\parskip}{6pt plus 2pt minus 1pt}
}
\setlength{\emergencystretch}{3em}  % prevent overfull lines
\providecommand{\tightlist}{%
  \setlength{\itemsep}{0pt}\setlength{\parskip}{0pt}}
\setcounter{secnumdepth}{0}
% Redefines (sub)paragraphs to behave more like sections
\ifx\paragraph\undefined\else
\let\oldparagraph\paragraph
\renewcommand{\paragraph}[1]{\oldparagraph{#1}\mbox{}}
\fi
\ifx\subparagraph\undefined\else
\let\oldsubparagraph\subparagraph
\renewcommand{\subparagraph}[1]{\oldsubparagraph{#1}\mbox{}}
\fi

%%% Use protect on footnotes to avoid problems with footnotes in titles
\let\rmarkdownfootnote\footnote%
\def\footnote{\protect\rmarkdownfootnote}

%%% Change title format to be more compact
\usepackage{titling}

% Create subtitle command for use in maketitle
\newcommand{\subtitle}[1]{
  \posttitle{
    \begin{center}\large#1\end{center}
    }
}

\setlength{\droptitle}{-2em}

  \title{}
    \pretitle{\vspace{\droptitle}}
  \posttitle{}
    \author{}
    \preauthor{}\postauthor{}
    \date{}
    \predate{}\postdate{}
  

\begin{document}

my\_num1 \textless{}- 2.33 my\_num1 my\_num2 \textless{}- 2.0 my\_num2
my\_num3 \textless{}- 2 my\_num3

my\_int1 \textless{}- 2L my\_int1 my\_int2 \textless{}- 2.0L my\_int2
my\_int3 \textless{}- 2.33L my\_int3

8 \textgreater{} 7 \# 判斷 8 是否???於 7 8 \textless{} 7 \# 判斷 8
是否???於 7 8 \textgreater{}= 7 \# 判斷 8 是否???於等於 7 8 \textless{}=
7 \# 判斷 8 是否???於等於 7 8 == 7 \# 判斷 8 是否等於 7 8 != 7 \# 判斷 8
是否不等於 7 7 \%in\% c(8, 7) \# 判斷 7 是否包含於???個 c(8, 7) 之中

first\_name \textless{}- ``Tony'' first\_name class(first\_name)

sys\_date \textless{}- Sys.Date() \# 系統???期 sys\_date \#
看起來跟???字相同 class(sys\_date)

sys\_time \textless{}- Sys.time() \# 系統時間 sys\_time \#
看起來跟???字相同 class(sys\_time)

my\_int1 \textless{}- 2L class(my\_int1) my\_int2 \textless{}- 2.0L
class(my\_int2) my\_int3 \textless{}- 2.33L class(my\_int3) class(TRUE)
class(FALSE) class(T) class(F) class(True) class(true) first\_name
\textless{}- `Tony' first\_name class(first\_name)

weather \textless{}- sample(c(``sunny'', ``rainy''), size = 1) weather
if (weather == ``sunny'')\{ print(``Running outdoors!'') \} else \{
print(``Working out in the gym!'') \}

my\_seq \textless{}- 1:10 for (i in my\_seq) \{ if (i \%\% 2 == 0) \{
print(paste(i, ``是偶數'')) \} else \{ print(paste(i, ``是奇數'')) \} \}

weather \textless{}- sample(c(``sunny'', ``rainy'', ``cloudy''), size =
1) weather if (weather == ``sunny'')\{ print(``Running outdoors!'') \}
else if (weather == ``cloudy'')\{ print(``Cycling!'') \} else \{
print(``Working out in the gym!'') \}

CHscore\textless{}-95 \#\#國???成績 ENscore\textless{}-55 \#\#英???成績
if(CHscore\textgreater{}=60)\{ if(ENscore\textgreater{}=60)\{
print(``全部及格'') \}else\{ print(``國???及格,英???再加油'') \}
\}else\{ if(ENscore\textgreater{}=60)\{
print(``英???及格,國???再加油'') \}else\{ print(``全部不及格'') \} \}

for (n in 1:10)\{ \#n為單???變數,1:10為需要逐???執???的參數向量
print(n) \}

x \textless{}- 0 while (x\textless{}=5) \{ print(x) x\textless{}-x+1 \}


\end{document}
